\ifx\isEmbedded\undefined

\documentclass[12pt,a4paper]{book}

\usepackage[utf8]{inputenc}
\usepackage[spanish]{babel}

\usepackage{amsmath}
\usepackage{amsfonts}
\usepackage{amssymb}
\usepackage{amsthm}
\usepackage{graphicx}
\linespread{1}	% double spaces lines

\parindent 1.5pt
\parskip 5pt  % Also, a bit of space between paragraphs

\newtheorem{teo}{Teorema}[chapter]
\newtheorem{defi}[teo]{Definición}
\newtheorem{lem}[teo]{Lema}
\newtheorem{prop}[teo]{Proposición}
\newtheorem{cor}[teo]{Corolario}
\newtheorem{ej}{Ejercicio}

\def\K{\mathbb{K}}
\def\N{\mathbb{N}}
\def\C{\mathbb{C}}
\def\R{\mathbb{R}}
\def\Z{\mathbb{Z}}
\def\Q{\mathbb{Q}}
\def\P{\mathcal{P}}
\def\F{\mathcal{F}}
\def\c{\mathfrak{c}}
\def\si{\mathrm{\;si\:}}
\def\a0{\aleph_0}
\def\GL{\mathrm{GL}}
\def\Ort{\mathrm{O}}
\def\Uni{\mathrm{U}}
\def\Ima{\mbox{Im}}
\def\Id{\mbox{Id}}
\def\sep{\,\vert\,}
\newcommand{\cl}[1]{\overline{#1}}
\newcommand{\bb}[1]{\mathbb{#1}}
\newcommand{\inn}[1]{{{#1} \in \N}}
\newcommand{\ini}[1]{{{#1} \in I}}
\newcommand{\serie}[2]{\sum_{{#1} = 1}^{#2}}
\newcommand{\seriei}[1]{\sum_{{#1} = 1}^{\infty}}


\begin{document}
\else
\fi
\section{Espacios métricos separables}
En esta sección veremos los espacios métricos separables, escencialmente, la separabilidad pone una restricción en que tan ''grande'' es el espacio métrico, no sólo en cardinalidad, sino también en su estructura.\\
Empezamos definiendo que es un conjunto denso:
\begin{defi}
Dado un espacio métrico $X$, y subconjuntos $D$ y $U$, decimos que $D$ es denso en $U$ si $\cl{D} \supseteq U$. Si tomamos a $U$ como $X$, tenemos que $X = \cl{D}$ y decimos simplemente que $D$ es denso.
\end{defi}
Ahora definimos separable:
\begin{defi}
Dado un espacio métrico separable $X$, decimos que es separable si existe un conjunto denso $D$ que además es contable.
\end{defi}
Los ejemplos obvios de espacios separables son $\R^n$ para todo $n$ natural, con $\Q^n$ como conjuntos densos.\\[0.5cm]
Ahora mostramos, como decíamos este resultado acerca de la cardinalidad de conjuntos separables.
\begin{prop}
Si $X$ es un espacio métrico separable entonces $\sharp X \leq \c$
\begin{proof}
Como $X$ es separable, significa que existe un conjunto $D \subseteq X$ tal que $ \cl{D} = X$ y que $\sharp D \leq \a0$, debido a que $D$ es denso significa que todo elemento en $X$ puede escribirse como sucesiones de $D$, por lo que existe una funcion inyectiva de $X$ a $D^{\N}$, lo que termina la demostración.
\end{proof}
\end{prop}
Como contraejemplo de que en el resultado anterior no vale la recíproca, cualquier conjunto con cardinalidad $\c$ y la métrica discreta resultará no separable.\\[0.5cm]
Veremos a continuación que la condición que un espacio sea separable no sólo restringe su cardinalidad, sino también la cantidad de abiertos que puede poseer y otras características en de éstos. Damos ahora algunas definiciones que luego vincularemos en un teorema.
\begin{defi}
Dado un espacio métrico $X$, una familia $\mathcal{B} = \{U_i\}_{i \in I}$ es una base de abiertos, si todos los elementos de la familia son subconjuntos abiertos de $X$ y además, dado un abierto $W$ de $X$, existe un subconjunto $J \subseteq I$ tal que $ W = \bigcup_{i \in J} U_i$.
\end{defi}
Por ejemplo en cualquier espacio métrico $X$, el conjunto $\mathcal{B} = \{ B_r(x)\sep x\in X, r \in \R\}$ forma una base, incluso $\mathcal{B} = \{ B_{\frac{1}{n}}(x) \sep x\in X,  n \in \N \}$, de cardinalidad aparentemente menor, también es base de abiertos.
\begin{defi}
Dado un espacio métrico $X$ y un conjunto $A \subseteq X$, una familia $\{W_j\}_{j \in J}$ es un cubrimiento de $A$ si $ \bigcup_{j \in J} W_j \supseteq A$.\\
Decimos que el cubrimiento es abierto (respectivamente cerrado) si todos los elementos de la familia lo son.\\
Un subcubrimiento es un subconjunto $I \subseteq J$ tal que la familia $\{W_j\}_{j \in I}$ sigue siendo un cubrimiento.
\end{defi}
\begin{defi}[Propiedad de Lindelöff]
Decimos que un espacio métrico $X$ cumple la propiedad de Lindelöff si para todo todo cubrimiento por abiertos de $X$, existe un subcubimiento contable.
\end{defi}
Ahora damos el teorema que vincula estas definiciones:
\begin{teo}
Dado un espacio métrico $X$, son equivalentes:
\begin{enumerate}
\item $X$ es separable.
\item Existe una base de abiertos de $X$ que es contable.
\item $X$ tiene la propiedad de Lindelöff.
\end{enumerate}
\begin{proof}
(1 $\Rightarrow$ 2) Si tenemos un conjunto $D$ contable y denso en $X$, entonces el afirmamos que el conjunto $ \mathcal{B} = \{ B_{\frac{1}{n}}(x) \sep n \in \N, x \in D\}$ es una base de abiertos contable. Es contable por la clara biyección entre $\mathcal{B}$ y $\N \times D$. Mostremos ahora que es una base. Si mostramos que puede generar todas las bolas abiertas, automáticamente resulta base, ya que todas las bolas abiertas son base. Ahora tomemos un punto $z \in X$ y consideremos para algún $r >0$, $B_r(z)$, si tomamos $y \in B_r(z)$ y $m_y \in \N$ tal que $B_{\frac{2}{m_y}}(y) \subseteq B_r(z)$, debido a que $D$ es denso, existe algún $x_y \in D$ tal que $x_y \in B_{\frac{1}{m_y}}(y)$, ahora podemos asegurar que $ y \in B_{\frac{1}{m_y}}(x_y) \subseteq B_{\frac{2}{m_y}}(y) \subseteq B_r(z)$, entonces  resulta claro que 
$$ B_r(z) = \bigcup_{y \in B_r(z)} B_{\frac{1}{m_y}}(x_y)$$
Y los uniendos son elementos de $\mathcal{B}$, por lo que ésta familia genera las bolas abiertas y por lo tanto genera todos los abiertos de $X$.\\
(2 $\Rightarrow$ 3) Si tenemos una base de abiertos $\mathcal{B}=\{U_n\}_\inn{n}$ y un cubrimiento por abiertos de $X$, $\{V_i\}_{i \in I}$, consideremos el subconjunto $\{U_{n_k}\}_\inn{k} \subseteq \mathcal{B}$, tal que para cada $U_{n_k}$ existe al menos un $i \in I$ tal que $U_{n_k} \subseteq V_i$, hagamos una elección de estos y llamemos para cada $n_k$, $i_k$ aquél que cumple esta condición. Ahora tenemos que:
$$ \bigcup_{k \in \N} V_{i_k} \subseteq \bigcup_\inn{k} U_{n_k}$$
Debido a que uno está incluído en el otro pero $\mathcal{B}$ es una base de abiertos, tenemos que para todo $j \in I$, $V_j = \bigcup\{U \in \mathcal{B} \sep U \subseteq V_j\} \subseteq \bigcup_\inn{k} U_{n_k}$, por lo que uniendo todo:
$$ X = \bigcup_{i \in I} V_i \subseteq \bigcup_\inn{k} U_{n_k}$$
Lo que muestra que la familia $V_{i_k}$ cubre a $X$, lo que muestra que es un subcubrimiento y además es contable.\\
(3 $\Rightarrow$ 1) Si para cada $n \in \N$ consideramos el conjunto de abiertos $\mathcal{B}_n = \{ B_{\frac{1}{n}}(x) \sep x \in X\}$, como éste cubre todo $X$, debido a la propiedad de Lindelöff, tenemos un subcubrimiento finito $\tilde{\mathcal{B}}_n = \{ B_{\frac{1}{n}}(x^{(n)}_k) \sep \inn{k}\}$. Afirmamos que el conjunto $D=\{x^{(n)}_k \sep \inn{k},\inn{n}\}$ es denso, además de ser claramente contable. Sea $\varepsilon >0$ y $y \in X$, podemos tomar $n \in \N$ tal que $1/n < \varepsilon$ y por lo tanto tenemos que $\{ B_{\frac{1}{n}}(x^{(n)}_k) \sep \inn{k}\}$ cubre $X$, así debe existir $k \in \N$ tal que $y \in B_{1/n}(x^{(n)}_k)$ entonces $d(x^{(n)}_k,y)<1/n < \varepsilon$. Por lo tanto $D$ es denso.
\end{proof}
\end{teo}
\ifx\isEmbedded\undefined
\end{document}
\else
\fi