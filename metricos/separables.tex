\ifx\isEmbedded\undefined

\documentclass[12pt,a4paper]{book}

\usepackage[utf8]{inputenc}
\usepackage[spanish]{babel}

\usepackage{amsmath}
\usepackage{amsfonts}
\usepackage{amssymb}
\usepackage{amsthm}
\usepackage{graphicx}
\linespread{1}	% double spaces lines

\parindent 1.5pt
\parskip 5pt  % Also, a bit of space between paragraphs

\newtheorem{teo}{Teorema}[chapter]
\newtheorem{defi}[teo]{Definición}
\newtheorem{lem}[teo]{Lema}
\newtheorem{prop}[teo]{Proposición}
\newtheorem{cor}[teo]{Corolario}
\newtheorem{ej}{Ejercicio}

\def\K{\mathbb{K}}
\def\N{\mathbb{N}}
\def\C{\mathbb{C}}
\def\R{\mathbb{R}}
\def\Z{\mathbb{Z}}
\def\Q{\mathbb{Q}}
\def\P{\mathcal{P}}
\def\F{\mathcal{F}}
\def\c{\mathfrak{c}}
\def\si{\mathrm{\;si\:}}
\def\a0{\aleph_0}
\def\GL{\mathrm{GL}}
\def\Ort{\mathrm{O}}
\def\Uni{\mathrm{U}}
\def\Ima{\mbox{Im}}
\def\Id{\mbox{Id}}
\def\sep{\,\vert\,}
\newcommand{\cl}[1]{\overline{#1}}
\newcommand{\bb}[1]{\mathbb{#1}}
\newcommand{\inn}[1]{{{#1} \in \N}}
\newcommand{\ini}[1]{{{#1} \in I}}
\newcommand{\serie}[2]{\sum_{{#1} = 1}^{#2}}
\newcommand{\seriei}[1]{\sum_{{#1} = 1}^{\infty}}


\begin{document}
\else
\fi

\section{Espacios métricos separables}
Dos resultados acerca de la cardinalidad de conjuntos separables.
\begin{prop}
Si $X$ es un espacio métrico separable entonces $\sharp X \leq \c$
\begin{proof}
Como $X$ es separable, significa que existe un conjunto $D \subseteq X$ tal que $ \overline{D} = X$ y que $\sharp D = \a0$, debido a que $D$ es denso significa que todo elemento en $X$ puede escribirse como sucesiones de $D$, por lo que existe una funcion inyectiva de $X$ a $D^{\N}$, lo que termina la demostración.
\end{proof}
\end{prop}
\begin{prop}
Si $X$ es separable y completo y además toda bola es infinita, entonces $\sharp X > \a0$
\begin{proof}
La demostración es bastante técnica y por eso la voy a obviar. Es fácil ver que $\sharp X > \a0$ debido a la completitud y usando que toda bola es infinita creando una sucesión de bolas cerradas encajadas, lo técnico es mostrar que el cardinal es exactamente $\c$. La idea intuitiva para esto, es que puede crearse una sucesión en ''binario'' de las bolas tomando una mitad de la primera y luego una mitad de la mitad y asi sucesivamente, creando una biyección entre $\{0,1\}^{\N}$ y $X$.
\end{proof}
\end{prop}


\ifx\isEmbedded\undefined
\end{document}
\else
\fi